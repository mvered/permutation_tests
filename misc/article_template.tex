% interacttfqsample.tex
% v1.05 - August 2017

\documentclass[]{interact}

\usepackage{epstopdf}% To incorporate .eps illustrations using PDFLaTeX, etc.
\usepackage[caption=false]{subfig}% Support for small, `sub' figures and tables
%\usepackage[nolists,tablesfirst]{endfloat}% To `separate' figures and tables from text if required

%\usepackage[doublespacing]{setspace}% To produce a `double spaced' document if required
%\setlength\parindent{24pt}% To increase paragraph indentation when line spacing is doubled
%\setlength\bibindent{2em}% To increase hanging indent in bibliography when line spacing is doubled

\usepackage[numbers,sort&compress]{natbib}% Citation support using natbib.sty
\bibpunct[, ]{[}{]}{,}{n}{,}{,}% Citation support using natbib.sty
\renewcommand\bibfont{\fontsize{10}{12}\selectfont}% Bibliography support using natbib.sty

\theoremstyle{plain}% Theorem-like structures provided by amsthm.sty
\newtheorem{theorem}{Theorem}[section]
\newtheorem{lemma}[theorem]{Lemma}
\newtheorem{corollary}[theorem]{Corollary}
\newtheorem{proposition}[theorem]{Proposition}

\theoremstyle{definition}
\newtheorem{definition}[theorem]{Definition}
\newtheorem{example}[theorem]{Example}

\theoremstyle{remark}
\newtheorem{remark}{Remark}
\newtheorem{notation}{Notation}
\usepackage{natbib}
\bibliographystyle{bib_style}

$if(highlighting-macros)$
$highlighting-macros$
$endif$

\begin{document}

\title{$title$}

\author{
\name{$author${a}\thanks{CONTACT $contact$}}
\affil{\textsuperscript{a}$affiliation$}
}

\maketitle

\begin{abstract}
Permutation test based U-statistics can offer finite sample guarantees on Type I and Type II error rates and are minimax optimal in a variety of contexts. We focus on the application of permutation test based \textit{U}-statistics for two-sample comparisons of medians as well as for independence testing in high dimensional multinomial problems. We attempt to validate the minimax framework for permutation test based \textit{U}-statistics proposed by Kim, Balakrishnan, and Wasserman \cite{ilmun_kim_minimax_2022}, by employing a series of simulations to investigate the performance of permutation test against other test statistics and compare Type I and Type II error rates under different sample sizes and different assumptions about the true underlying data distribution.
\end{abstract}

\begin{keywords}
permutation tests, \textit{U}-statistics, non-parametric testing, minimax optimality
\end{keywords}

$body$

\bibliography{$bibliography$}

\end{document}
